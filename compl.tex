% !TeX spellcheck = it_IT
\title{Complementi di Algoritmi}
\author{Massimo Perego}
\date{}

\documentclass[11pt]{report}
\usepackage{graphicx} 
\usepackage{amsmath}
\usepackage{amssymb}
\usepackage{amsfonts}
\usepackage[hidelinks]{hyperref}
\usepackage[autostyle, english = american]{csquotes}
\usepackage[parfill]{parskip}
\MakeOuterQuote{"}

\usepackage{amsthm}

\renewcommand{\chaptername}{Capitolo}
\renewcommand{\contentsname}{Indice}
\renewcommand{\proofname}{Dimostrazione}

\theoremstyle{plain}
\newtheorem{theorem}{Teorema}[section]
\newtheorem{fact}{Fatto}[section]
\newtheorem{lemma}{Lemma}[section]
\newtheorem{cor}{Corollario}[section]

\newtheorem*{theorem*}{Teorema}
\newtheorem*{fact*}{Fatto}
\newtheorem*{lemma*}{Lemma}
\newtheorem*{cor*}{Corollario}

\renewcommand{\O}{\mathcal{O}}

\newcommand{\I}{\mathcal{I}}
\newcommand{\X}{\mathcal{X}}
\newcommand{\C}{\mathcal{C}}
\newcommand{\polred}{\preceq_p}
\newcommand{\equivpol}{\equiv_p}

\begin{document}
    \maketitle
    \tableofcontents
    \newpage
    
    \chapter{Complessità Computazionale}
    
    % !TeX spellcheck = it_IT
% !TeX root = ../../compl.tex
\section{Riducibilità polinomiale tra problemi di decisione}

Un \textbf{problema di decisione} $X$ su $\{0,1\}$ è una coppia $(\I, q)$ dove
\begin{itemize}
    \item $\I \subseteq \{0,1\}^\ast$ è l'insieme delle descrizioni (come stringhe binarie) delle \textbf{istanze valide del problema}

    \item $q: \I \rightarrow \{0,1\}$ è la \textbf{funzione di decisione}, che associa ogni istanza $I \in \I$ al suo valore,  ovvero $q(I) = 1$ se positiva o $q(I) = 0$ se negativa
\end{itemize}

Un algoritmo $A$ risolve un problema $X = (\I, q)$ se restituisce il valore corretto della funzione $q$ su ogni istanza $I \in \I$.

Indichiamo con $|I|$ la \textbf{lunghezza della codifica} in bit di un'istanza $I \in \I$. In genere, non specifichiamo il modo con il quale un'istanza viene rappresentata come una stringa di bit, ma resta inteso che le istanze vengano rappresentate secondo rappresentazioni "ragionevoli" (e.g. lista di adiacenza per un grafo).

Se $A$ è un algoritmo che risolve il problema $X = (\I, q)$, indichiamo con $T_A (I)$ il tempo di calcolo di $A$ quando l'input è $I \in \I$. Definiamo ora
$$ T_A (n) = \max \left\{T_A (I) \mid |I| \leq n \right\}\}$$
ovvero il massimo dei tempi di calcolo per istanze $I \in \I$ di lunghezza al più $n$. Diciamo che $A$ risolve $X$ in tempo polinomiale se esiste un intero $k$ tale che $T_A(n) = \O \left(n^k\right)$.

Un algoritmo ha accesso a un \textbf{oracolo} per un problema $X = (\I, q)$ quando può ottenere in tempo unitario il valore $q(I)$ per istanze arbitrarie $I \in \I$.

Un problema $Y$ è \textbf{polinomialmente riducibile} a un altro problema $X$, denotato con $Y \polred X$ quando
\begin{enumerate}
    \item Esiste un algoritmo che risolve $Y$ in tempo polinomiale con accesso a un oracolo per $X$

    \item L'oracolo viene interrogato una sola volta e l'algoritmo termina fornendo la risposta dell'oracolo
\end{enumerate}

Questa nozione di riducibilità polinomiale è anche nota come \textit{riducibilità secondo Karp}. Esiste anche una nozione più generale, nota come riducibilità secondo Turing, dove l'algoritmo può accedere all'oracolo un numero polinomiale di volte.

Equivalentemente, la riducibilità polinomiale (secondo Karp) può essere vista come "l'input di $Y$ può essere trasformato in tempo polinomiale in input di $X$" (per poi risolvere il problema in tempo unitario grazie all'oracolo, coincide con la prima definizione).

Se $Y \polred X$ ed esiste un algoritmo che risolve $X$ in tempo polinomiale, allora esiste un algoritmo che risolve $Y$ in tempo polinomiale (in quanto la polinomialità è robusta rispetto alla composizione). Viceversa, se $Y \polred X$ e non esiste un algoritmo che risolve $Y$ in tempo polinomiale, allora non esiste un algoritmo che risolve $X$ in tempo polinomiale. Intuitivamente, $Y \polred X$ significa che il problema $Y$ non è \textit{più difficile} del problema $X$. Se $X \polred Y$ e $Y \polred X$, ovvero $X$ e $Y$ sono polinomialmente riducibili l'uno con l'altro, allora scriviamo $X \equivpol Y$.

\subsection{Esempi di riducibilità tra problemi}

Sia $G = (V,E)$ un grafo semplice (non diretto, non pesato, senza loop e archi multipli). Un \textbf{insieme indipendente} (Independent Set) in $G$ è un sottoinsieme $S \subseteq V$ di vertici non adiacenti, ovvero $\forall i,j \in S$, $(i,j) \notin E$. Una \textbf{copertura} (Vertex Cover) di $G$ è un sottoinsieme $S' \subseteq V$ di vertici tali che ogni arco di $G$ ha almeno un estremo in $S'$, ovvero $\forall (i,j) \in E$ esiste $k \in S'$ tale che $k = i$ o $k = j$. \\

\begin{fact}
    \label{f:isvc}
    Sia $G = (V,E)$ un grafo semplice. Allora $S$ è un insieme indipendente (soluzione per Independent Set) se e solo se $S' = V \setminus S$ è una copertura (soluzione per Vertex Cover).
\end{fact}
\begin{proof}
    Sia $S$ un insieme indipendente e sia $S' = V \setminus S$. Allora dato un qualsiasi $(i,j) \in E$, deve valere $i \notin S$ oppure $j \notin S$. Quindi $i \in S'$ oppure $j \in S'$, da cui ne segue che $S'$ è una copertura di $G$.

    Viceversa, sia $S'$ una copertura di $G$ e $S = V \setminus S'$. Allora dati $i,j \in S$ arbitrari $(i,j) \notin E$, altrimenti allora $S'$ non sarebbe una copertura; da questo otteniamo che $S$ è un Independent Set.
\end{proof}

Sia \textsc{Independent Set} il problema di decisione le cui istanze sono coppie $I = (G,k)$ dove $G$ è un grafo semplice e $k$ è un intero. La funzione di decisione $q_{IS}$ è tale che $q_{IS} (I) = 1$ se e solo se $G$ contiene un insieme indipendente di taglia almeno $k$.

Le istanze del problema \textsc{Vertex Cover} sono le stesse di Independent Set, ma la funzione di decisione $q_{VC}$ è tale che $q_{VC} (I) = 1$ se e solo se $G$ contiene una copertura di taglia al più $k$. 

Conseguenza immediata del Fatto \ref{f:isvc} è che Independent Set e Vertex Cover sono polinomialmente riducibili l'uno con l'altro. \\

\begin{cor}
    Independent Set $\equivpol$ Vertex Cover
\end{cor}

La relazione $\equivpol$ gode di transitività. \\

\begin{fact}
    Se $Z \polred Y$ e $Y \polred X$, allora $Z \polred X$.
\end{fact}
\begin{proof}
    Dato un oracolo per $X$ possiamo risolvere un'istanza di $Z$ nel modo seguente: eseguiamo l'algoritmo per risolvere $Z$ usando l'oracolo per $Y$, ma ogni volta che dovrebbe essere invocato l'oracolo di $Y$, questo viene simulato eseguendo l'algoritmo per risolvere $Y$, il quale usa l'oracolo per $X$ (ancora una volta, la polinomialità è robusta rispetto alla composizione).
\end{proof}

Introduciamo ora il problema di decisione \textsc{Set Cover}: le istanze sono della forma $I = (U, \mathcal{S}, k)$ dove $U$ è un insieme finito, $\mathcal{S} \equiv \left\{S_1, \dots, S_m \right\}$ è una collezione di sottoinsiemi di $U$ e $k$ è un intero. Allora $q_{SC} (I) = 1$ se e solo se esistono al più $k$ sottoinsiemi in $\mathcal{S}$ tali che la loro unione sia tutto $U$.\\

\begin{fact}
    Vertex Cover $\polred$ Set Cover
\end{fact}
\begin{proof}
    Supponiamo di avere un oracolo per Set Cover e definiamo un algoritmo polinomiale per Vertex Cover.

    Data un'istanza $I = (G,k)$ di Vertex Cover con $G = (V,E)$ si può costruire un'istanza $I'$ di Set Cover dove $U \equiv E$ e $\mathcal{S} \equiv \left\{S_i \mid i \in V \right\}$, dove $S_i$ è l'insieme degli archi di $G$ incidenti sul vertice $i$ (ottenibile in tempo lineare partendo dalla descrizione di $G$). Quindi ogni $u \in U$ è contenuto in esattamente due elementi di $\mathcal{S}$ (un lato ha due estremità).

    Verifichiamo ora che $U$ è l'unione di al più $k$ insiemi $S_1, \dots, S_{|V|}$ se e solo se $G$ ha una copertura con vertici di taglia al più $k$.

    Supponiamo che l'unione di $S_{i_1}, \dots, S_{i_r}$ sia $U$, con $r \leq k$. Allora, ogni arco in $G$ è incidente a uno dei vertici $i_1, \dots, i_r$. Quindi $\left\{i_1, \dots, i_r\right\}$ è una copertura con vertici di taglia al più $k$. Viceversa, se $\left\{i_1, \dots, i_r\right\}$ è una copertura con vertici di taglia al più $k$ allora la gli insiemi corrispondenti $S_{i_1}, \dots, S_{i_r}$ hanno $U$ come unione.

    Quindi possiamo definire un algoritmo che implementa la funzione di decisione $q_{VC}$ per Vertex Cover nel modo seguente: data un'istanza di $I$ Vertex Cover, l'algoritmo costruisce in tempo polinomiale un'istanza $I'$ di Set Cover nel modo descritto sopra. Quindi chiama l'oracolo un'unica volta per ottenere $q_{SC} (I')$ che viene restituita in output.
\end{proof}

L'universo di Set Cover corrisponde agli archi da coprire, ogni insieme corrisponde agli archi che incidono su un nodo, quindi selezionare uno degli insiemi nella collezione corrisponde a scegliere un nodo come parte della copertura.

Così come possiamo vedere Set Cover come una generalizzazione di Vertex Cover, introduciamo ora Set Packing come generalizzazione di Independent Set.

Le istanze di \textsc{Set Packing} sono le stesse di Set Cover, ovvero $I = (U, \mathcal{S}, k)$ dove $U$ è un insieme finito, $\mathcal{S} \equiv \left\{S_1, \dots, S_m \right\}$ è una collezione di sottoinsiemi di $U$ e $k$ è un intero. la funzione di decisione $q_{SP}$ è tale che $q_{SP} = 1$ se e solo se esistono almeno $k$ sottoinsiemi in $\mathcal{S}$ i quali sono a due a due disgiunti. In altre parole, nessun elemento dell'universo compare in più di uno degli insiemi scelti.

Anche se Set Packing è apparentemente più generale di Independent Set, si può dimostrare che i due problemi sono equivalenti. \\

\begin{fact}
    Independent Set $\equivpol$ Set Packing.
\end{fact}
\begin{proof}
    Data un'istanza $\mathcal{S} \equiv \left\{S_1, \dots, S_m\right\}$ di Set Packing costruiamo in tempo polinomiale un grafo $G = (V,E)$ dove $V \equiv \left\{v_S \mid S \in \mathcal{S} \right\}$ e $\left(v_S, v_T\right) \in E$ se e solo se $S \cap T \not \equiv \emptyset$ (un grafo con un vertice per ogni insieme del problema originale, ed è presente un arco tra due vertici solo se i relativi insiemi non sono disgiunti, ovvero se hanno elementi in comune). Allora ogni insieme indipendente in $G$ corrisponde a un packing della stessa taglia.

    Viceversa, dato un grafo $G = (V,E)$ possiamo costruire in tempo polinomiale la collezione $\mathcal{S} \equiv \left\{S_i \mid i \in V \right\}$ dove $S_i$ è l'insieme degli archi di $G$ incidenti su $i$. Allora ogni packing in $\mathcal{S}$ corrisponde a un insieme indipendente in $G$ della stessa taglia.
\end{proof}

In molte discipline bisogna spesso risolvere problemi di ottimizzazione combinatoria vincolata. In questi problemi si cerca un assegnamento di valori per un insieme di variabili discrete in modo da soddisfare un dato insieme di vincoli. In astratto, problemi di questo tipo sono formulati come problemi di soddisfacibilità su variabili booleane.

Sia $\X$ un insieme di variabili booleane $x_1, \dots, x_n$. Un assegnamento di valori di verità su $\X$ è una funzione $\pi: \X \rightarrow \{0,1\}$. Un letterale $\ell_i$ è la variabile $x_i$ o la sua negazione $\bar x_i$. Una clausola $C = \ell_{i_1} \vee \dots \vee \ell_{i_k}$ è una disgiunzione di letterali. Un assegnamento $\pi$ soddisfa una clausola $C$ se e solo se c'è almeno un letterale della forma $\ell_i = x_i$ e $\pi(x_i) = 1$ oppure $\ell_j = \bar x_j$ e $\pi(x_j) = 0$ (in altre parole, \href{https://it.wikipedia.org/wiki/Forma_normale_congiuntiva}{\texttt{una formula in CNF}}).

Sia \textbf{SAT} il problema di decisione le cui istanze $I$ sono insiemi di clausole $\C$ su un insieme $\X$ di variabili booleane. Allora $q(I) = 1$ se e solo se esiste un assegnamento $\pi: \X \rightarrow \left\{0,1\right\}$ tale che soddisfi tutte le clausole in $\C$ (in realtà sarebbe CNFSAT, SAT è la versione su formule generiche, ma ogni formula è riducibile a una CNF equisoddisfacibile, ovvero soddisfacibile se e solo se lo era anche la formula originaria, in tempo polinomiale, senza però mantenere l'equivalenza semantica).

Una versione ridotta di SAT è \textbf{3-SAT}, le cui istanze sono insiemi di clausole ciascuna contenente esattamente 3 letterali. Esempio di istanza di 3-SAT:
\begin{itemize}
    \item $\X \equiv \left\{x_1, x_2, x_3, x_4\right\}$

    \item $\C \equiv \left\{\left(\bar x_1 \vee x_2 \vee \bar x_3 \right), \left(\bar x_2 \vee x_3 \vee x_4 \right), \left(x_1 \vee x_2 \vee \bar x_4 \right) \right\}$
\end{itemize}
Un assegnamento che soddisfa tutte le clausole è $\pi(x_i) = 0$ per $i = 1, \dots, 4$. \\

\begin{theorem}
    3-SAT $\polred$ Independent Set.
\end{theorem}
\begin{proof}
    Una definizione equivalente di 3-SAT: $\C$ è soddisfacibile se e solo se è possibile scegliere esattamente un letterale in ciascuna clausola in modo tale che tra i letterali scelti non compaiano simultaneamente $x_i$ e $\bar x_i$ per nessuna delle variabili $x_i \in \X$.

    Data un'istanza $I = \C$ di 3-SAT, costruiamo un'istanza $I' = \left(G, |\C|\right)$ di Independent Set tale che $q_{3SAT} (I) = q_{IS} (I')$. Sia $k = |\C|$. Il grafo $G = (V,E)$ ha $3k$ vertici, uno per ogni letterale in $\C$. I tre vertici corrispondenti ai letterali di ciascuna clausola formano una clique (ognuno è connesso agli altri due, per fare in modo che un solo letterale per clausola venga scelto). Inoltre, per ogni $x_i \in \X$, se $x_i$ e $\bar x_i$ sono letterali in clausole distinte allora c'è un arco in $G$ fra i vertici corrispondenti (per fare in modo che solo un letterale o la sua negazione venga scelto). Si noti che possiamo costruire $G$ a partire da $\C$ in tempo polinomiale rispetto a $|\C|$.

    Per costruzione di $G$, l'unico modo per ottenere un insieme indipendente di taglia $k$ nel grafo $G$ è quello di scegliere un nodo in ciascuna delle $k$ clique in modo che non ci siano archi fra il nodo scelto in una clique e nodi scelti appartenenti a un'altra clique. Per costruzione, questo avviene se e solo se è possibile scegliere esattamente un letterale in ciascuna clausola in modo tale che tra i letterali scelti non compaiano simultaneamente $x_i$ e $\bar x_i$ per nessuna delle variabili $x_i \in \X$.

    Possiamo quindi costruire un algoritmo che risolve 3-SAT in tempo polinomiale usando un oracolo per Independent Set: l'algoritmo riceve un'istanza $I$ di 3-SAT, costruisce l'istanza corrispondente $I'$ di Independent Set e accede all'oracolo ottenendo $q_{IS} (I')$, per poi restituire $q_{IS} (I')$.
\end{proof}
    
    
    
    
    
    
    
    
    
\end{document}
