% !TeX spellcheck = it_IT
% !TeX root = ../../compl.tex
\section{$k$-Means}
\label{prob: km}

Consideriamo il problema di partizionare un insieme finito $\X \subset \R^d$ di punti in $k > 1$ cluster. La similarità tra punti può essere misurata come distanza Euclidea, dato che siamo in $\R^d$. Identifichiamo ogni cluster $i \in \left\{1, \dots, k\right\}$ tramite il relativo centro $\bm c_i \in \R^d$ (non è necessario che $\bm c_i \in \X$) e assegniamo ogni punto $\bm x \in \X$ al centro più vicino (rispetto alla distanza Euclidea).

Il costo di un punto all'interno di una partizione (clustering) $\C = \left\{\bm c_1, \dots, \bm c_k\right\}$ è $\phi (\C, \bm x) = \min_{i = 1, \dots, k} \|\bm x - \bm c_i \|^2$.

Il costo della partizione $\C$ è $\Phi (\C) = \sum_{\bm x \in \X} \phi (\C, \bm x)$.

Si noti che ogni punto "costa" quanto il quadrato della distanza dal centro più vicino. Il $k$-clustering ottimale $\C^\ast$ è una qualsiasi scelta di centri (partizione) tale che minimizzi il costo, ovvero
$$ \C^\ast = \arg \min_{\bm c_1, \dots, \bm c_k \in \R^d} \Phi \left(\bm c_1,  \dots, \bm c_k\right) $$
Non è detto che l'insieme di centri ottimali sia unico. Denotiamo con $\Opt(\X)$ il costo di $\C^\ast$.

\boxProb{$k$-means}
{insieme $\X \subset \R^d$, parametro $k>1$}
{qualsiasi $\C \subset \R^d$ con $|C| = k$ tale che $\Phi(\C) = \Opt(\X)$}

Il problema è triviale per $k=1$, essendoci un unico centro $\bm c^\ast$ che minimizza il costo, il quale corrisponde al centroide dell'insieme $\X$
$$ \bm c^\ast = \arg \min_{\bm c \in \R^d} \sum_{\bm x \in \X} \|\bm x - \bm c\|^2 = \frac{1}{|\X|} \sum_{\bm x \in \X} \bm x $$

Questo può essere dimostrato notando che $F(\bm c) = \sum_{\bm x \in \X} \|\bm x - \bm c\|^2$ è una funzione convessa con minimo nel punto in cui $\bm c$ è il centroide. Questo implica che $\C^\ast$ è formato dai centroidi dei cluster che lo compongono.

Il problema di $k$-means assume implicitamente che i punti in $\X$ siano campionati da $k$ distribuzioni Gaussiane sferiche $\Nc (\bm \mu_i, \sigma_i^2 I)$ per $i = 1, \dots, k$ le quali medie $\bm \mu_1, \dots, \bm \mu_k$ sono i centri e le quali varianze $\sigma_1^2, \dots, \sigma_k^2$ sono upper bound per il costo ottimale
$$ \bm \mu_i = \arg \min_{\bm c} \Ex \left[\|\bm X - \bm c\|^2 \right], \ \text{dove } \bm X \sim \Nc \left(\bm \mu_i, \sigma_i^2 I \right) \ \text{ e } \ \Ex \left[\Phi\left(\bm \mu_1, \dots, \bm \mu_k\right)\right] \leq \sum_{i = 1}^k \sigma_i^2 $$

Risolvere il problema in $\R^d$ è $\NP$-hard anche per $k=2$ (quando $d = 2n$). Di conseguenza, il migliore algoritmo esatto per risolvere $k$-means si basa su
\begin{enumerate}
    \item Enumerare tutte le $k^{|\X|}$ partizioni di $\X$ in $k$ componenti

    \item Calcolare i centroidi $\C = \left\{\bm c_1, \dots, \bm c_k\right\}$ per i $k$ elementi della partizione

    \item Calcolare il costo $\Phi (\C)$ della partizione
\end{enumerate}

Indichiamo con $\X(i)$ il sottoinsieme di punti di $\X$ tali che hanno $\bm c_i$ come centro più vicino
$$ \X(i) = \left\{\bm x \in \X \mid \arg \min_{j = 1, \dots, k} \|\bm x - \bm c_j \|^2 = i \right\}$$

L'algoritmo seguente è l'euristica più comune per risolvere $k$-means.

\begin{algorithm}
    \label{alg:lloyd}
    \caption{Algoritmo di Lloyd}
    \KwInput{Insieme finito $X \subset \R^d$, parametro $k$ t.c. $1 < k < |\X|$}
    Scegliere uniformemente a caso $k$ punti da $\X$\;
    \Repeat{$\bm c_1, \dots, \bm c_k$ rimangono invariati}{
        \For{$\bm x \in \X$}{
            Assegna $\bm x$ al cluster $C_i$, con $i = \arg \min_{j = 1, \dots, k} = \|\bm x - c_j\|^2$\;
        }
        \For{$i = 1, \dots, k$}{
            $\displaystyle \bm c_i \leftarrow \frac{1}{|C_i|} \sum_{\bm x \in \X(i)} \bm x$ \tcp*[r]{$\bm c_i$ è il centroide di $C_i$}
        }
    }
\end{algorithm}

Il tempo per ogni iterazione dell'algoritmo è nell'ordine di $\O(nkd)$. Si possono usare proiezioni casuali per mappare $\X$ a $\R^N$ con $N = \Theta (\ln n)$, risultando in un valore di \Opt\ moltiplicato per una costante. Questo riduce il tempo per iterazione a $\O (nk \ln n)$. Il caso peggiore per quanto riguarda il numero di iterazioni dell'algoritmo, purtroppo, è $2^{\Omega (\sqrt{n})}$.

Anche se funziona bene in casi concreti, l'algoritmo di Lloyd non approssima \Opt\ per nessuna costante. \\

\begin{theorem}
    \label{theo:inapprox-lloyd}
    Per ogni $a > 1$ esistono istanze 1-dimensionali $\X \subset \R$ di $k$-means con $k = 3$ per le quali l'algoritmo di Lloyd restituisce una partizione $\C$ tale che $\Phi (\C) \geq a \cdot \Opt$ con probabilità arbitrariamente vicina a 1.
\end{theorem}
\begin{proof}
    Scelto un $a > 1$ e sia $\X$ di dimensione $n$ tale che $n-2$ punti sono spaziati equamente all'interno del segmento $[0,1]$, mentre i due punti rimanenti (outliers) sono posizionati a $2 \sqrt{an}$ e $3 \sqrt{an}$
    \begin{center}
        \begin{tikzpicture}[
    % styles for dots and triangles
    dot/.style={circle, fill=black, inner sep=0pt, minimum size=3pt},
    triangle/.style={regular polygon, regular polygon sides=3, draw=red, fill=none, thick, inner sep=0pt, minimum size=6pt}
    ]
    % arrow
    \draw [->] (0,0) -- (11,0);

    % points
    \node[dot] at (0.2,0) {};
    \node[dot] at (0.4,0) {};
    \node[dot] at (0.6,0) {};
    \node[dot] at (0.8,0) {};
    \node[triangle] at (0.7,0) {};
    \node[dot] at (1.0,0) {};
    \node[dot] at (1.2,0) {};
    \node[triangle] at (1.1,0) {};
    \node[dot] at (1.4,0) {};
    \node[dot] at (1.6,0) {};
    \node[dot] at (1.8,0) {};

    % outliers
    \node[dot] at (4.8,0) {};
    \node[triangle] at (7.3,0) {};
    \node[dot] at (9.8,0) {};

\end{tikzpicture}
    \end{center}

    La probabilità che l'algoritmo di Lloyd non scelga entrambi gli outlier come centri iniziali è calcolata come segue: ci sono $\binom{n}{3}$ modi possibili per scegliere 3 punti su $n$ e $n-2$ modi di scegliere 3 punti tali che tra questi ci siano i due outlier. Quindi la probabilità è
    $$ p_n = 1 - \frac{n-2}{\binom{n}{3}} = 1 - \frac{(n-3)! 6 (n - 2)}{n!} ) 1 - \frac{6}{n (n - 1)} $$

    Consideriamo quindi il caso sfavorevole in cui l'algoritmo sceglie inizialmente al più uno degli outlier. In questo caso, l'algoritmo termina con almeno due centri all'interno di $[0,1]$ è al più un centro in $\frac{5}{2} \sqrt{an}$. Il costo $\Phi (\C)$ di questa partizione $\C$ è almeno $\frac{an}{2}$, mentre il costo del cluster ottimo (due centri sugli outlier e il rimanente su $1/2$) è $\Opt = \frac{n-2}{4}$.

    Quindi, $\Phi(\C) / \Opt = \Omega (a)$. Al crescere di $n$, ovvero per $n \rightarrow \infty$, abbiamo che $p_n \rightarrow 1$, con la conseguenza che il caso sfavorevole accade con probabilità arbitrariamente alta.
\end{proof}

Ora mostriamo come, se i centri si muovono, il valore di $\Phi$ decresce strettamente e può farlo al più $\O (k^n)$ volte, ovvero il numero di possibili partizioni in $k$ componenti di $\X$ con $|\X| = n$.\\

\begin{lemma}
    Se in una iterazione un qualsiasi centro viene mosso, allora $\Phi$ decresce strettamente.
\end{lemma}
\begin{proof}
    Faremo uso del fatto seguente. Per ogni $C \subset \R^d$ finito e per ogni $\bm c \in \R^d$
    \[ \sum_{\bm x \in C} \|\bm x - \bm c\|^2 = |C|\|\bm c - \bm \mu \|^2 + \sum_{\bm x \in C} \|\bm x - \bm \mu \|^2  \tag*{($\dag$)}\]
    dove $\bm \mu$ è il centroide di $C$.

    Siano $C_1, \dots, C_k$ e $\bm c_1, \dots, \bm c_k$ rispettivamente cluster e centri all'inizio di una iterazione (Linea 2) e siano $C_1', \dots, C_k'$ e $\bm c_1', \dots, \bm c_k'$ cluster e centri al termine dell'iterazione (Linea 7). Sia
    $$ \psi \left(C_1, \dots, C_k, \bm c_1, \dots, \bm c_k \right) = \sum_{i = 1}^k \sum_{\bm x \in C_i} \|\bm x - \bm c_i \|^2 $$

    Si noti che $\psi \left(C_1, \dots, C_k, \bm c_1, \dots, \bm c_k \right) \geq \psi \left(C_1', \dots, C_k', \bm c_1, \dots, \bm c_k \right)$ dato che Linea 4 assegna ogni punto al centro più vicino.

    Ora, se $\bm c_i' \neq \bm c_i$ per qualche $i$, allora
    $$ \psi \left(C_1', \dots, C_k', \bm c_1, \dots, \bm c_k \right) > \psi \left(C_1', \dots, C_k', \bm c_1', \dots, \bm c_k' \right) $$

    Per dimostrarlo, ricordando che $\bm c_i'$ è il centroide di $C_i'$,
    $$ \sum_{\bm x \in C_i'} \| \bm x - \bm c_i \|^2 \stackrel{(\dag)}{=} |C'| \|\bm c_i - \bm c_i'\|^2 + \sum_{\bm x \in C_i'} \| \bm x - \bm c_i' \|^2 > \sum_{\bm x \in C_i'} \| \bm x - \bm c_i' \|^2  $$
    usando $(\dag)$ nel primo passaggio e $\bm c_i \neq \bm c_i'$ nel secondo. Di conseguenza
    \begin{align*}
        \Phi \left(C_1, \dots, C_k\right)  & = \psi \left(C_1, \dots, C_k, \bm c_1, \dots, \bm c_k \right) & \\
        & > \psi \left(C_1', \dots, C_k', \bm c_1', \dots, \bm c_k' \right) & = \Phi \left(C_1', \dots, C_k' \right)
    \end{align*}
\end{proof}

Questo implica anche il seguente risultato.\\

\begin{theorem}
    L'algoritmo di Lloyd termina in al più $k^{|\X|}$ iterazioni per qualsiasi input $(\X, k)$.
\end{theorem}
\begin{proof}
    Si può notare che $\Phi$ è una funzione applicata alla partizione corrente $\left\{C_1, \dots, C_k\right\}$, tale partizione può assumere al più $k^n$ valori distinti. Inoltre, l'algoritmo non termina solo se l'iterazione corrente ha modificato la partizione. Dato che $\Phi$ può solo decrescere quando la partizione viene modificata, l'algoritmo deve terminare in al più $k^n$ iterazioni.
\end{proof}

% end k-means.pdf