% !TeX spellcheck = it_IT
% !TeX root = ../../compl.tex
\section{Il problema del Coupon Collector}

Il problema del coupon collector è definito come segue: sia $X_1, X_2, \dots$ una sequenza di variabili casuali indipendenti e uniformemente distribuite su $n$ valori distinti $a_1, \dots, a_n$
$$ \Pr \left(X_t = a_i \right) = \frac{1}{n}, \quad i = 1, \dots, n, \ \  t \geq 1 $$

Calcolare $\Ex[N]$, dove $N = \min \left\{k \mid \left(\forall i \leq n\right) \left(\exists t \leq k\right) X_t = a_i \right\}$. In altre parole, $N$ è il minimo numero di realizzazioni $x_1, \dots, x_k$ sufficienti a osservare ciascun $a_i$ almeno una volta.

Il nome \textit{coupon collector} deriva dal problema di collezionare tutti gli $n$ possibili coupon contenuti in prodotti da acquistare (per esempio, scatole di cereali), dove ogni scatola contiene uno qualsiasi dei buoni premio con probabilità uniforme.

Un problema equivalente è il seguente: supponiamo che a ogni lancio, una pallina cade con probabilità uniforme in una fra $n$ possibili scatole. Quante palline devo lanciare in media affinché ce ne sia almeno una in ogni scatola?

Un'applicazione concreta del coupon collector è la seguente: supponiamo di voler sapere gli identificativi degli $n$ router attraversati da una sequenza di pacchetti. Mentre non c'è abbastanza spazio in un pacchetto per memorizzare tutti gli $n$ identificativi, è facile memorizzare in un pacchetto l'identificativo di un router a caso tra quelli attraversati. Ci si chiede allora quanti pacchetti servono in media per ottenere gli identificativi di tutti gli $n$ router.

Per analizzare il problema, suddividiamo $X_1, X_2, \dots$ in $n$ blocchi di lunghezze $N_1, \dots, N_n$, dove $N_i$ è il numero di estrazioni aggiuntive che servono per ottenere l'$i$-esimo valore distinto avendone già osservati $i-1$. Quindi
$$ N = \sum_{i=1}^n N_i $$

Le variabili casuali $N_1, \dots, N_n$ sono tutte Geometriche. In particolare, quando $i-1$ valori distinti sono già stati osservati, la probabilità di osservarne uno nuovo è
$$ p_i = 1 - \frac{i - 1}{n} = \frac{n - i + 1}{n} $$

Infatti, $p_1 = 1$ e questo implica $N_1 = 1$ deterministicamente, ovviamente.

Ricordando che il valore atteso di una Geometrica di parametro $p_i$ è $\Ex[N_i] = \frac{1}{p_i}$ (Lemma \ref{lemma:vadg}), per linearità del valore atteso abbiamo
$$ \Ex [N] = \sum_{i=1}^n \Ex [N_i] = \sum_{i=1}^n \frac{1}{p_i} = \sum_{i=1}^n \frac{n}{n - i + 1} = n \sum_{i=1}^n \frac{1}{i} = n \ln n + \Theta (n) $$
dove l'ultima uguaglianza vale perché la somma armonica $1 + \frac{1}{2} + \dots + \frac{1}{n}$ è asintotica a $\ln n + \Theta(1)$.

% end coupon.pdf