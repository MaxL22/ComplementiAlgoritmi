% !TeX spellcheck = it_IT
% !TeX root = ../../compl.tex
\section{Reservoir Sampling}

Si consideri il problema di mantenere una struttura dati che, a ogni istante di tempo, contenga $k$ elementi estratti a caso con probabilità uniforme da uno stream di elementi in ingresso. In particolare, vogliamo sviluppare un algoritmo che soddisfi il seguente invariante: per ogni $t \geq k$, ognuno dei primi $t$ elementi dello stream è contenuto nella struttura dati con probabilità pari a $\frac{k}{t}$.

Per esempio, vogliamo stimare le percentuali delle varie tipologie di oggetti (libri, elettronica, abbigliamento, \dots) venduti su Amazon in un dato lasso di tempo. Se ogni oggetto venduto è campionato con la stessa probabilità, allora la distribuzione delle tipologie nel campiona sarà tendenzialmente uguale a quella nello stream.

Studiamo il problema nel modello streaming: a ogni istante di tempo $t = 1,2, \dots$ l'algoritmo può accedere soltanto al $t$-esimo elemento $x_t$ dello stream. Chiediamo inoltre che l'algoritmo lavori in spazio $\Theta(k)$.

Il seguente algoritmo soddisfa tutte le proprietà richieste.
\begin{algorithm}[h!]
    \caption{Reservoir Sampling}
    \KwInput{Intero $k$}
    $R = \emptyset$ \tcp*[r]{Inizializza la riserva}
    \For{$t = 1,2, \dots$}{
        Leggi il prossimo elemento $x_t$ nello stream\;
        \eIf{$t \leq k$}{
            Aggiungi $x_t$ a $R$\;
        }{
            Con probabilità $\frac{k}{t}$, sostituisci un elemento a caso in $R$ con $x_t$
        }
    }
\end{algorithm}

Nel caso in cui lo stream avesse lunghezza nota $N$, potremmo aggiungere alla riserva ogni elemento dello stream in modo indipendente con probabilità $\frac{k}{N}$. Questo garantirebbe la proprietà che ogni elemento dello stream è contenuto nella riserva con la stessa probabilità, ma il numero di elementi effettivamente inseriti nella riserva potrebbe essere maggiore o minore di $k$. \\

\begin{theorem}
    Sia $R_t$ il contenuto della riserva dopo che sono stati osservati i primi $t$ elementi dello stream. Per ogni $t \geq k$ vale: $\Pr \left(x_i \in R_t \right) = \frac{k}{t}$ per ogni $i \leq t$.
\end{theorem}

Per la dimostrazione useremo più volte il fatto che, per ogni coppia di eventi $A,B$ tale che $\Pr(B) > 0$ vale $\Pr(A \cap B) = \Pr(B) \Pr(A \mid B)$.

\begin{proof}
    La dimostrazione è per induzione su $t \geq k$:
    \begin{itemize}
        \item \textbf{Base:} $t = k$. Allora $\Pr\left(x_i \in R_t\right) = 1 = \frac{k}{t}$ dato che $t = k$

        \item \textbf{Step:} Fissato $t \geq k$, assumiamo l'ipotesi induttiva
        $$ P\left(x_i \in R_t \right) = \frac{k}{t}, \quad \forall i \leq t $$
        e dimostriamo
        $$ \Pr\left(x_i \in R_{t+1}\right) = \frac{k}{t+1}, \quad \forall i \leq t+1 $$
        Se $i = t+1$, allora vale la tesi per costruzione (riga 7 dell'algoritmo). Se invece $i \leq t$, dato che $x_i \in R_{t+1}$ implica $x_i \in R_t$, abbiamo che $\Pr\left(x_i \in R_{t+1}\right) = \Pr \left(x_i \in R_{t+1}, \ x_i \in R_t \right)$. Possiamo quindi scrivere
        \begin{align*}
            \Pr\left(x_i \in R_{t+1}\right) & = \Pr \left(x_i \in R_{t+1}, \ x_i \in R_t \right) \\
            & = \Pr \left(x_i \in R_t \right) \Pr \left(x_i \in R_{t+1} \mid x_i \in R_t \right) \\
            & = \frac{k}{t} \cdot \Pr \left(x_i \in R_{t+1} \mid x_i \in R_t \right) && \text{(per IH)}
        \end{align*}

        Ora si osservi che, dato $x_i \in R_t$, abbiamo che $x_i \notin R_{t+1}$ implica $x_{t+1} \in R_{t+1}$. Quindi possiamo scrivere
        \begin{align*}
            \Pr \left(x_i \in R_{t+1} \mid x_i \in R_t \right) & = 1 - \Pr \left(x_i \notin R_{t+1} \mid x_i \in R_t \right) \\
            & = 1 - \Pr\left(x_i \notin R_{t+1}, \ x_{t+1} \in R_{t+1} \mid x_i \in R_t \right) \\
            & = 1 - \Pr \left(x_{t+1} \in R_{t+1} \mid x_i \in R_t\right) \Pr \left(x_i \notin R_{t+1} \mid x_{t+1} \in R_{t+1}, \ x_i \in R_t \right) \\
            & = 1 - \frac{k}{t+1} \frac{1}{k} \\
            & = \frac{t}{t+1}
        \end{align*}
        dove
        $$ \Pr \left(x_{t+1} \in R_{t+1} \mid x_i \in R_t \right) = \Pr \left(x_{t+1} \in R_{t+1}\right) = \frac{k}{t+1} $$
        per costruzione dell'algoritmo e
        $$ \Pr\left(x_i \notin R_{t+1} \mid x_{t+1} \in R_{t+1}, \ x_i \in R_t \right) = \frac{1}{k} $$
        dato che $x_i$ ha probabilità uniforme di essere selezionato dalla riserva per far posto a $x_{t+1}$. Quindi
        $$ \Pr \left(x_i \in R_{t+1}\right) = \frac{k}{t} \frac{t}{t+1} = \frac{k}{t+1} $$
        che conclude la dimostrazione.
    \end{itemize}
\end{proof}

% end reservoir.pdf