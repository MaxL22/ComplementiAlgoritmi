% !TeX spellcheck = it_IT
% !TeX root = ../../compl.tex
\section{Estrattore di von Neumann}

Un estrattore di causalità è una funzione che trasforma una sorgente non perfettamente casuale in una completamente casuale. Il più semplice estrattore è quello ideato da John von Neumann e risponde alla domanda: \textit{come è possibile usare una moneta truccata per simulare dei lanci di moneta non truccata?}

Più precisamente, avendo una moneta con probabilità sconosciuta $0 < p < 1$ di restituire testa ogni volta che viene lanciata, la si vuole usare per simulare una sequenza di lanci di una moneta equa, ovvero con probabilità $1/2$ di restituire testa.

Siano $X_1, X_2, \dots$ le variabili casuali Bernoulliane indipendenti con $\Pr(X_t = 1) = p$ che modellano i lanci della moneta truccata. Consideriamo le coppe $(X_1, X_2), (X_3, X_4), \dots$ e notiamo che i valori possibili per ogni coppia sono:

\begin{tabular}{c l}
    $(0,0)$ & con probabilità $(1 - p)^2$ \\
    $(1,1)$ & con probabilità $p^2$ \\
    $(0,1)$ e $(1,0)$ & con probabilità $p(1 - p)$
\end{tabular}

Quindi, per ogni coppia $(X_{2k-1}, X_{2k})$ gli eventi $(X_{2k-1}, X_{2k}) = (0,1)$ e $(X_{2k-1}, X_{2k}) = (1,0)$ sono equiprobabili e forniscono la sequenza di lanci desiderata.

\begin{algorithm}[hbt!]
    \caption{Estrattore di von Neumann}
    \KwInput{Sequenza di lanci $X_1, X_2, \dots$}
    \For{$k = 1,2, \dots$}{
        \If{$X_{2k-1} \neq X_{2k}$ \tcp*[r]{Controlla se è una coppia utile}}{
            \eIf{$X_{2k-1} = 1$}{
                Print "Testa" \;
            }{
                Print "Croce" \;
            }
        }
    }
\end{algorithm}

Praticamente, lancia la moneta truccata finché non ottieni due valori diversi di seguito: se il primo dei due è testa, il lancio "equo" è testa, croce altrimenti. Questo funziona in quanto la probabilità che esca $(0,1)$ o $(1,0)$ è la stessa.

Possiamo ora calcolare quanti lanci di moneta truccata servono in media per simulare un lancio di moneta non truccata. Data una sequenza $Z_1, Z_2, \dots$ di variabili Bernoulliane indipendenti tali che $\Pr (Z_k = 1) = q$ per $k \geq 1$, la variabile casuale geometrica $G$ è definita come $G = \min \left\{k = 1,2, \dots \mid Z_k = 1\right\}$.

Chiaramente $\Pr (G = 1) = q$ e $\Pr (G = n) = (1 - q)^{n-1}q$ per ogni $n > 1$. Non è difficile dimostrare che $\Ex [G] = \frac{1}{q}$ (Lemma \ref{lemma:vadg}).

Consideriamo ora la sequenza $Z_1, Z_2, \dots$ di variabili Bernoulliane indipendenti tali che
$$ Z_k = \begin{cases}
    1 & X_{2k-1} \neq X_{2k} \\
    0 & \text{altrimenti}
\end{cases}$$

%TODO:  controlla meglio il calcolo, perché 2?
Per quanto detto prima, $\Pr(Z_k = 1) = 2p(1-p)$. Sia $G$ la variabile geometrica associata alla sequenza delle $Z_k$. Quindi il numero medio di lanci che mi servono è
$$ 2 \Ex [G] = \frac{1}{p (1 - p)} $$

% end vonNeumann.pdf